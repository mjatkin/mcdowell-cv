%% The MIT License (MIT)
%%
%% Copyright (c) 2015 Daniil Belyakov
%%
%% Permission is hereby granted, free of charge, to any person obtaining a copy
%% of this software and associated documentation files (the "Software"), to deal
%% in the Software without restriction, including without limitation the rights
%% to use, copy, modify, merge, publish, distribute, sublicense, and/or sell
%% copies of the Software, and to permit persons to whom the Software is
%% furnished to do so, subject to the following conditions:
%%
%% The above copyright notice and this permission notice shall be included in all
%% copies or substantial portions of the Software.
%%
%% THE SOFTWARE IS PROVIDED "AS IS", WITHOUT WARRANTY OF ANY KIND, EXPRESS OR
%% IMPLIED, INCLUDING BUT NOT LIMITED TO THE WARRANTIES OF MERCHANTABILITY,
%% FITNESS FOR A PARTICULAR PURPOSE AND NONINFRINGEMENT. IN NO EVENT SHALL THE
%% AUTHORS OR COPYRIGHT HOLDERS BE LIABLE FOR ANY CLAIM, DAMAGES OR OTHER
%% LIABILITY, WHETHER IN AN ACTION OF CONTRACT, TORT OR OTHERWISE, ARISING FROM,
%% OUT OF OR IN CONNECTION WITH THE SOFTWARE OR THE USE OR OTHER DEALINGS IN THE
%% SOFTWARE.

% The font could be set to Windows-specific Calibri by using the 'calibri' option
\documentclass[]{mcdowellcv}

% For mathematical symbols
\usepackage{amsmath}

% Set applicant's personal data for header
\name{Matthew Atkin}
\address{linkedin.com/in/matthewatkin \linebreak github.com/mjatkin}
\contacts{0428 848 873 \linebreak matthewjamesatkin@gmail.com}

\begin{document}

	% Print the header
	\makeheader
	
	% Print the content
	\begin{cvsection}{Career Objective}
		\begin{cvsubsection}{}{}{}
			I am an enthusiastic final year computer science \& computer engineering double major seeking a software engineering graduate role at Google.
			I have excellent leadership and teamwork skills, from my 5 years of experience in hospitality and numerous university group projects over 4 years.
			I am experienced in programming, especially in C/C++, and fluent in common algorithms and data structures, from several university and hobby projects.
		\end{cvsubsection}
	\end{cvsection}

	\begin{cvsection}{Education}
		\begin{cvsubsection}{Melbourne, VIC}{RMIT University}{Mar 2016 -- Present}
			\begin{itemize}
				\item \textbf{Double Major}: Bachelor of Engineering (Computer and Network Engineering) (Honours)/Bachelor of Computer Science (GPA: 3.3) (Expected Dec 2020) 
				\item \textbf{Computer Science Coursework}: Operating Systems, Algorithms \& Data Structures, Network Programming, Artificial Intelligence, Web Programming, Database Concepts, Software Engineering, Graphics Programming
				% Operating Systems, Algorithms \& Data Structures, Network Programming, Artificial Intelligence, Web Programming, Database Concepts, Software Engineering, Graphics Programming
				\item \textbf{Electrical Eng. Coursework}: Circuit Design, Signal Processing, Embedded System Programming, Network Engineering, Digital System Design, Computer Architecture, RTOS Programming
				% Circuit Design, Signal Processing, Embedded System Programming, Network Engineering, Digital System Design, Computer Architecture, RTOS Programming
			\end{itemize}
		\end{cvsubsection}
	\end{cvsection}

	\begin{cvsection}{Relevent Projects}
		\begin{cvsubsection}{UNIX Clone For ARMv7-A}{Hobby Project}{Jan 2020 -- Current}
			\begin{itemize}
				\item Developed a UNIX-like kernel targeting the ARM Cortex-A15 core in QEMU.
				\item Incorporated exception handling, virtual memory, multitasking, UART driver and keyboard handling.
				\item Diagnosed and debugged problems within the kernel using GDB.
				\item Extensively made use of Git and Github to more easily manage development between multiple developers.
				\item \underline{Utilised}: C++ Programming, ARM Assembly Programming, QEMU, GDB, Git
			\end{itemize}
		\end{cvsubsection}

		\begin{cvsubsection}{Real Time Traffic Controller}{RMIT University}{Aug 2019 -- Nov 2019}
			\begin{itemize}
				\item Designed a real time traffic control system for two four-way intersections separated by a train crossing.
				\item Created a custom PCB with NeoPixel LEDs to represent the intersections and train crossing.
				\item Utilised multiple DE-10 Nano nodes and networked IPC to control each intersection/crossing separately.
				\item Implemented fail safe procedures in the system, e.g. if the train controller node goes offline, the adjacent intersection will not route traffic towards the train crossing.
				\item \underline{Utilised}: C++ Programming, Circuit Design, DE-10 Nano, QNX RTOS Programming, Git
			\end{itemize}
		\end{cvsubsection}
\iffalse
-Worked in a team of four to develop a real time traffic controller system for two four-way intersections separated by a train crossing.
-The intersections were represented by a custom PCB our team designed and created, utilising NeoPixel LEDs as our traffic lights and push buttons as our pedestrian buttons. 
-The system itself was controlled by four separate micro-controller nodes (DE-10 Nano) running the QNX real time operating system, one for each intersection, one for the train crossing and one more as the controlling/watchdog system. 
-Each node was heavily multi threaded and networked IPC was utilised to communicate between nodes. 
-Node failure was accounted for in the design, with fail safe mechanisms employed to stop potentially dangerous scenarios, e.g. if the train crossing controller node went offline, the adjacent intersections would not route traffic towards the crossing.
-The lights realistically change, transitioning from green to yellow to red in real time, with the transitions being triggered by either timers, or the pedestrian buttons depending on the mode.
-The controller node featured a command line interface, which allowed individual control over each other nodes mode, light configuration and timing. 
\fi
		\begin{cvsubsection}{Soundcard Device Controller}{RMIT University}{Mar 2019 -- Jun 2019}
			\begin{itemize}
				\item Designed an ISA device controller for the Sound Blaster 16 sound card for the DE-10 Nano FPGA.
				\item Created a completely custom riser card to interface the ISA bus connector to the DE-10 Nano GPIO pins.
				\item Diagnosed problems with the riser card and ISA device controller using an oscilloscope.
				\item Configured memory mapped IO between the FPGA and hard processor system and created a device driver.
				\item \underline{Utilised}: C Programming, Verilog Programming, Intel Quartus Prime, Oscilloscope, DE-10 Nano
			\end{itemize}
		\end{cvsubsection}
\iffalse
-In a pair, designed and developed an ISA device controller for the Sound Blaster 16 sound card for the DE-10 Nano SoC.
-A custom riser card was created to interface the ISA bus connector on the sound card to the DE-10 Nano GPIO pins.
-The controller itself was written in Verilog and deployed on the DE-10 Nano FPGA, this consisted of write, read and configuration registers, as well as logic to match the specific timing requirements of the ISA standard.
-Intel Quartus Prime was utilised to memory map the controller registers to the hard processor system, allowing us to write and read from the card transparently through a basic custom driver.
\fi

\iffalse
		\begin{cvsubsection}{Arduino LED Matrix}{}{}
			\begin{itemize}
				\item Designed and implemented a custom 8x8 LED matrix board with device driver for an Arduino Microcontroller.
				\item \underline{Utilised}: C Programming, Arduino Microcontroller, Circuit Design
			\end{itemize}
		\end{cvsubsection}
\fi

	\end{cvsection}

	\begin{cvsection}{Employment}
		\begin{cvsubsection}{Melbourne, VIC}{KFC -- Front of House}{Dec 2013 -- Jan 2019}
			\begin{itemize}
				\item Delivered consistent and quality service as a cohereant team while in high pressure and stressful times.
				\item Trained and mentored new team members to become confident and team orientated in their service.
				\item Achieved employee of the month on several occasions due to a consistent and reliable work ethic.
			\end{itemize}
		\end{cvsubsection}
	\end{cvsection}

	\begin{cvsection}{Other Employment}
		\begin{cvsubsection}{RMIT ECE Club}{Committee Member}{Mar 2019 -- Present}
			\begin{itemize}
				\item Elected as 2019 secretary and 2020 vice president of the Electrical \& Computer Engineering Club.
				\item Managed club social media and email, as well as consistently engaged in club decision making \& logistics.
				\item Something about leadership
			\end{itemize}
		\end{cvsubsection}

		\begin{cvsubsection}{RMIT Peer Mentoring}{Student Mentor}{Mar 2018 -- Present}
			\begin{itemize}
				\item Completed three semesters of student mentoring for programming and computing theory based subjects.
				\item Provided academic coaching as well as advice on class scheduling, time management, and study skills.
				\item Something about adaptability
			\end{itemize}
		\end{cvsubsection}
	\end{cvsection}

	\begin{cvsection}{Skills}
		\begin{cvsubsection}{Technical}{}{}
			\begin{itemize}
				%\item \textbf{Software:} (\emph{proficient}): C, C++, Unix, Git, Verilog (\emph{familiar}): Java, Python, MATLAB, JavaScript, OpenGL
				\item 4+ years programming experience through university courses and hobby projects, with focus on C/C++.
				\item Data structures
				\item Proficiency in Unix/Linux usage and architecture through frequent use of them and Unix clone development.
			\end{itemize}
		\end{cvsubsection}
		\begin{cvsubsection}{Leadership \& Teamwork}{}{}
			\begin{itemize}
				\item s
				\item s
				\item s
			\end{itemize}
		\end{cvsubsection}
		\begin{cvsubsection}{Versitility \& Adaptability}{}{}
			\begin{itemize}
				\item m
				\item s
				\item s
			\end{itemize}
		\end{cvsubsection}
	\end{cvsection}

	\begin{cvsection}{References}
		\begin{cvsubsection}{}{}{}
			References Available Upon Request.
		\end{cvsubsection}
	\end{cvsection}

\end{document}
